\documentclass{article}
\usepackage[utf8]{inputenc}
\usepackage[greek,english]{babel}
\usepackage{alphabeta}
\usepackage{graphicx}
\usepackage{amsmath}  
\usepackage{placeins}
\usepackage{empheq}
\usepackage{float} 
\usepackage{multicol} 
\usepackage[svgnames]{xcolor}
\usepackage{listings}
\usepackage[margin=1.5in]{geometry} 
\addto\captionsenglish{% Replace "english" with the language you use
  \renewcommand{\contentsname}%
    {Περιεχόμενα}%
}

\lstset{language=R,
    basicstyle=\small\ttfamily,
    stringstyle=\color{DarkGreen},
    otherkeywords={0,1,2,3,4,5,6,7,8,9},
    morekeywords={TRUE,FALSE},
	frame = single, 
    deletekeywords={data,frame,length,as,character},
    keywordstyle=\color{blue},
    commentstyle=\color{DarkGreen},
}  


\title{Προγραμματιστικά Εργαλεία και Τεχνολογίες για Επιστήμη Δεδομένων \\
Πρόβλημα "time-travel"} 
\author{Μύρων Σαμψάκης-Μπακόπουλος\\  
 \\	 
Μεταπτυχιακό Πρόγραμμα Επιστήμης Δεδομένων και Μηχανικής Μάθησης\\ 
Εθνικό Μετσόβιο Πολυτενχείο}
\date{Νοέμβριος 2019}

\begin{document}

%\begin{titlepage}

%\maketitle
%\includegraphics{figs/ntua.png}
%\end{titlepage}  
%\newpage
\begin{titlepage}
%
\begin{flushleft}
\mbox{
\begin{minipage}[c]{0.2\textwidth}
   \includegraphics[width=2.cm]{figs/ntua.png}
\end{minipage}
\begin{minipage}[c]{30em}
   \raggedright
   \sc \bf
   {\large Εθνικό Μετσόβιο Πολυτεχνείο} \\
   Σχολή Ηλεκτρολόγων Μηχανικών-Μηχανικών Ηλεκτρονικών Υπολογιστών \\ 
   Διατμηματικό Μεταπτυχιακό Πρόγραμμα\\
   Μηχανικής Μάθησης και Επιστήμης Δεδομένων
\end{minipage}
}
\end{flushleft}

\vskip  3cm

\begin{center}
{\large \bf 
Εξόρυξη Γνώσης από Δεδομένα \\~\\
Chicago Crime Rate}

\vskip  2cm\
\vskip  1cm
{\large \bf Μύρων Σαμψάκης-Μπακόπουλος }\\ 
\vskip  0.2cm
{\large \bf Δημήτρης Λαμπράκης }\\ 
\vskip  0.2cm
{\large \bf Σπύρος Πούρος }
\vskip  0.5cm  


\vskip  2cm
Αθήνα, Φεβρουάριος 2020
\end{center}

\end{titlepage} 

\tableofcontents 

\newpage
\section{Εισαγωγή} 

Σκοπός της ερασίας αυτής είναι η εφαρμογή τεχνικών εξόρυξης γνώσης από το dataset Chicago Crime Rate που δίνεται από το BigQuery της Google.

Το dataset αυτό αποτελείται από δηλώσεις εγκλημάτων που γίναν στην περιοχή του Σικάγου κατά περίπου τα τελευταία 20 χρόνια. Πέραν του είδους του εγκλήματος, περιλαμβάνει πληροφορίες όπως η ώρα και η μέρα που έγινε, η περιοχή, εάν υπήρξε σύλληξη ή όχι, και διάφορα άλλα. Οι συνολικές καταχωρήσεις ανέρχονται περίπου στις 7 εκατομμύρια, καθιστώντας το αρκετά μεγλάλο.

Τα εργαλεία που χρησιμοποιήθηκαν...

\section{Data Augmentation}

\enddocument 
 